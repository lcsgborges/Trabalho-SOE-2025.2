
\begin{abstract}
%\boldmath
O resumo deve ressaltar o objetivo, o método, os resultados e as conclusões do documento.
A ordem e a extensão destes itens dependem do tipo de resumo (informativo ou indicativo) e do
tratamento que cada item recebe no documento original. 
Não separe o texto do resumo em parágrafos.

% Não temos dados o suficiente para realmente escrever um abstract, então vai ficar assim provavelmente até a entrega final - R


\end{abstract}
% IEEEtran.cls defaults to using nonbold math in the Abstract.
% This preserves the distinction between vectors and scalars. However,
% if the conference you are submitting to favors bold math in the abstract,
% then you can use LaTeX's standard command \boldmath at the very start
% of the abstract to achieve this. Many IEEE journals/conferences frown on
% math in the abstract anyway.

% no keywords
