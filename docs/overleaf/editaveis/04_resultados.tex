\section{Resultados Experimentais}\label{sec:resultados}

Os experimentos deverão validar o funcionamento do protótipo desenvolvido, comparando o que se espera dele com o que foi possível alcançar.
Explique os experimentos definidos, seguido de uma análise crítica dos resultados esperados e obtidos.
Em caso de divergências, aponte as possíveis causas. 
Explique claramente tudo o que foi feito.

Serão esperados os seguintes resultados para cada ponto de controle:

\begin{itemize}
    \item \textbf{PC1:} proposta do projeto, sem resultados práticos;
    \item \textbf{PC2:} funcionamento básico de cada parte fundamental, mostrando com quaisquer linguagens de programação que é possível conectar estas partes ao Raspberry Pi;
    \item \textbf{PC3:} refinamento do protótipo em linguagem C/C++;
    \item \textbf{PC4:} refinamento do protótipo em linguagem C/C++.
\end{itemize}
Fazendo uma analogia do projeto com a montagem de um quebra-cabeças, o PC1 corresponderia à escolha do quebra-cabeças, o PC2 seria a disposição de todas as peças sobre a mesa, e os PCs 3 e 4 seriam a montagem do quebra-cabeças.


A partir do PC2, os grupos deverão apresentar em sala de aula o funcionamento atualizado do sistema, e aproveitar os resultados documentados nos PCs para compôr esta Seção na entrega final.
Desta maneira, os pontos de controle indicam com clareza se o trabalho do grupo está adiantado ou atrasado em relação à Entrega Final\footnote{Desenvolvendo bem os quatro PCs, o grupo poderá chegar à entrega final com pouco trabalho por fazer.}.

A Tabela \ref{tab:pontuacao} apresenta a pontuação dada a cada uma das Seções e Subseções na avaliação final deste trabalho, bem como os pontos de controle onde elas serão pré-avaliadas.

\begin{table}[!htpb]
% increase table row spacing, adjust to taste
\renewcommand{\arraystretch}{1.3}
\caption{Avaliações deste trabalho}
\label{tab:pontuacao}
\centering
\begin{tabular}{ccc}
\hline
\textbf{Seção} & \textbf{Pontuação final} & \textbf{Pré-avaliação} \\
\hline 
\textit{Abstract} & 1 & ---\\
\ref{sec:intro}. Introdução & 1 & PC1 \\
\ref{subsec:hardware}. \textit{Descrição de} Hardware & 2 & PC2, PC3 e PC4\\
\ref{subsec:software}. \textit{Descrição de} Software & 3 & PC2, PC3 e PC4\\
\ref{sec:resultados}. \textit{Resultados Experimentais} & 2 & PC2, PC3 e PC4 \\
\ref{sec:conclusoes}. \textit{Conclusões} & 1 & --- \\
\hline
\textbf{Total} & 10 \\
\hline
\end{tabular}
\end{table}